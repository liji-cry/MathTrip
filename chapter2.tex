\chapter{容斥原理}
\section*{引言}
本章内容主要来自Brualdi <<组合数学>> 第6章(容斥原理及应用)的内容及对其的拓展与思考.

\section{容斥原理及其对偶形式}
设$S$是对象的有限集合, $P_1, P_2, \cdots, P_m$ 是 $S$ 的对象所涉及的 $m$ 个性质,并设
\[ A_i = \{ x : x \text{ 属于 } S \text{ 且 } x \text{ 具有性质 } P_i \} \quad (i = 1, 2, \cdots, m). \]

\begin{theorem}[容斥原理的对偶形式]
    集合 $S$ 中不具有性质 $P_1, P_2, \cdots, P_m$ 的对象个数由下面的交错表达式给出:
    \begin{align*}
          & |\overline{A}_1 \cap \overline{A}_2 \cap \cdots \cap \overline{A}_m|                                                         \\
        = & |S| - \sum |A_i| + \sum |A_i \cap A_j| - \sum |A_i \cap A_j \cap A_k| + \cdots + (-1)^m |A_1 \cap A_2 \cap \cdots \cap A_m|.
    \end{align*}
\end{theorem}

\begin{proof}
    显然, 不具备$P_1, P_2, \cdots, P_m$中任何一条性质的对象对等式右边的贡献为$1$, 该贡献源于$|S|$.

    对具备$n\geq 1$条性质的对象, 我们证明它对等式右边的贡献为$0$. 它对集合$|S|$, $\sum |A_i|$, $\sum |A_i \cap A_j|$, $\cdots$的贡献依次为: $1$, $\binom{n}{1}$, $\binom{n}{2}$, $\cdots$. 由推论\ref{exstl1}得其在等式右边贡献和为
    $$\binom{n}{0} - \binom{n}{1} + \binom{n}{2} - \binom{n}{3} + \cdots + (-1)^n \binom{n}{n}=0.$$
    即证.
\end{proof}

\begin{theorem}[容斥原理]
    集合 $S$ 中至少具有性质 $P_1, P_2, \cdots, P_m$ 之一的对象个数由下式给出:
    \begin{align*}
         & |A_1 \cup A_2 \cup \cdots \cup A_m| \\=&\sum |A_i| - \sum |A_i \cap A_j| + \sum |A_i \cap A_j \cap A_k| - \cdots + (-1)^{m+1} |A_1 \cap A_2 \cap \cdots \cap A_m|.
    \end{align*}
\end{theorem}
\begin{proof}
    由德摩根律
    $$\overline{A_1 \cup A_2 \cup \cdots \cup A_m} = \overline{A}_1 \cap \overline{A}_2 \cap \cdots \cap \overline{A}_m.$$

    因此,
    \begin{align*}
          & |A_1 \cup A_2 \cup \cdots \cup A_m|                                                                                        \\=&|S|-\overline{A_1 \cup A_2 \cup \cdots \cup A_m}\\
        = & |S|-\overline{A}_1 \cap \overline{A}_2 \cap \cdots \cap \overline{A}_m                                                     \\
        = & \sum |A_i| - \sum |A_i \cap A_j| + \sum |A_i \cap A_j \cap A_k| - \cdots + (-1)^{m+1} |A_1 \cap A_2 \cap \cdots \cap A_m|.
    \end{align*}
\end{proof}

\section{Möbius反演}
考虑一个有限偏序集$(X,\leq)$. 设 $\mathcal{F}(X)$ 是满足只要 $x \leq y$ 就有 $f(x, y) = 0$ 的所有实值函数
\[ f : X \times X \rightarrow \mathbb{R} \]
的集合,于是 $f(x, y)$ 只在 $x \leq y$ 时可能不等于 0. 我们如下定义 $\mathcal{F}(X)$ 中两个函数 $f$ 和 $g$ 的卷积 $h = f * g$:

\[ h(x, y) = \begin{cases}
        \sum_{z : x \leq z \leq y} f(x, z) g(z, y) & \text{若 } x \leq y, \\
        0                                          & \text{其他.}
    \end{cases} \]
\begin{lemma}\label{juanjijiehel}
    卷积满足结合律. 即$f * (g * h) = (f * g) * h \quad (f, g, h \in \mathcal{F}(X))$.
\end{lemma}
\begin{proof}证明$x\leq y$的情况.
    \begin{align*}
        f*(g*h)(x,y)= & \sum_{z:x\leq z\leq y}f(x,z)\cdot (g*h)(z,y)                        \\
        =             & \sum_{z:x\leq z\leq y}f(x,z)\sum_{z':z\leq z'\leq y}g(z,z')h(z',y)  \\
        =             & \sum_{z:x\leq z\leq y}\sum_{z':z\leq z'\leq y}f(x,z)g(z,z')h(z',y)  \\
        =             & \sum_{z':x\leq z'\leq y}\sum_{z:x\leq z\leq z'}f(x,z)g(z,z')h(z',y) \\
        =             & \sum_{z':x\leq z'\leq y}(f*g)(x,z')\cdot h(z',y)                    \\
        =             & (f * g) * h(x,y).
    \end{align*}
    即证.
\end{proof}

我们考虑$\mathcal{F}(X)$中三种特殊的函数:
\begin{itemize}
    \item $\delta$函数: $$\delta(x, y) =
              \begin{cases}
                  1 & \text{若 } x = y \\
                  0 & \text{其他}
              \end{cases}$$
              对所有的$f\in \mathcal{F}(X)$, 有$\delta*f=f*\delta=f$, 故称$\delta$为卷积下的恒等函数.
    \item $\zeta$函数: $$\zeta(x, y) =
              \begin{cases}
                  1 & \text{若 } x \leq y \\
                  0 & \text{其他}
              \end{cases}$$
              $\zeta$函数是偏序集$(X,\leq)$的一种表示.
    \item Möbius函数$\mu$: $\zeta$函数在卷积下的逆函数.
\end{itemize}

\begin{lemma}
    设 $f$ 是 $\mathcal{F}(X)$ 中的函数,对 $X$ 中的所有 $y$ 满足 $f(y, y) \neq 0$. 则$f$在卷积下存在逆函数.
\end{lemma}

\begin{proof}
    如下递归地定义$\mathcal{F}(X)$ 中的函数$g$:

    首先, 由于$f(y, y) \neq 0$, 可以定义 $$g(y, y) = \frac{1}{f(y, y)} \quad (y \in X).$$

    然后, $$g(x, y) = -\frac{1}{f(y, y)} \sum_{z:x\leq z < y} g(x, z) f(z, y) \quad (x < y).$$

    于是, 不难观察到 $$\sum_{z:x\leq z \leq y} g(x, z) f(z, y) = \delta(x, y) \quad (x \leq y)$$

    即证$g$是$f$在卷积意义下的左逆函数: $g*f=\delta$.

    同理可以证明$f$存在卷积意义下的右逆函数: $f*h=\delta$. 由引理\ref{juanjijiehel}, $$g = g * \delta = g * (f * h) = (g * f) * h = \delta * h = h.$$

    即证, $g$是$f$的逆函数.
\end{proof}

因此, 由于$\mu*\zeta=\delta$, 我们得到
$$\sum_{z : x \leq z \leq y} \mu(x, z) \zeta(z, y) = \sum_{z : x \leq z \leq y} \mu(x, z)=\delta(x, y) \quad (x \leq y).$$

由上式可知对所有的$x\in X$, 有$\mu(x,x)=1$. 以及$$\mu(x, y) = -\sum_{z : x \leq z < y} \mu(x, z) \quad (x < y)$$

利用$\mu$是$\zeta$的右逆函数也可以得到对偶的类似结果.

\begin{theorem}[Möbius反演]
    设 $(X, \leq)$ 是偏序集且有最小元 $0$。设 $\mu$ 是它的Möbius函数, 并设 $F$ : $X \rightarrow \mathbb{R}$ 是定义在 $X$ 上的实值函数. 设函数 $G$ : $X \rightarrow \mathbb{R}$ 是如下定义的函数:
\[ G(x) = \sum_{z : z \leq x} F(z) \quad (x \in X) \]

那么
\[ F(x) = \sum_{y : y \leq x} G(y) \mu(y, x) \quad (x \in X). \]
\end{theorem}

\begin{proof}
    \begin{align*}
\sum_{y : y \leq x} G(y) \mu(y, x) &= \sum_{y : y \leq x} \sum_{z : z \leq y} F(z) \mu(y, x) \\
&= \sum_{y : y \leq x} \mu(y, x) \sum_{z : z \in X} \zeta(z, y) F(z) \\
&= \sum_{z : z \in X} \sum_{y : y \leq x} \zeta(z, y) \mu(y, x) F(z) \\
&= \sum_{z : z \in X} \left( \sum_{y : z \leq y \leq x} \zeta(z, y) \mu(y, x) \right) F(z) \\
&= \sum_{z : z \in X} \delta(z, x) F(z) \\
&= F(x).
\end{align*}即证.
\end{proof}

\section{Möbius反演的应用}