\chapter{二项式系数}
\section*{预备知识}
基本组合数公式:
\[\binom{n}{k} = \binom{n}{n-k}.\]

帕斯卡公式:
\[\binom{n}{k} = \binom{n-1}{k}+\binom{n-1}{k-1}.\]

\section{二项式定理}
\begin{theorem}[二项式定理]
    设\(n\)是正整数.对所有的\(x\)和\(y\), 有\[(x + y)^n = \sum_{k=0}^{n} \binom{n}{k} x^{n-k} y^k.\]
\end{theorem}
\begin{proof}
    法1:乘积展开,对项\(x^{n-k}y^k\),选择\(k\)个因子为\(y\)而剩下\(n-k\)个因子为\(x\)的组合数.

    法2:归纳法.\(n=1\)时等式成立.

    假设等式对\(n\)成立,对于\(n+1\):

    \begin{align*}
(x + y)^{n+1} &= (x + y) \left( \sum_{k=0}^{n} \binom{n}{k} x^{n-k} y^k \right) \\
&= x \left( \sum_{k=0}^{n} \binom{n}{k} x^{n-k} y^k \right) + y \left( \sum_{k=0}^{n} \binom{n}{k} x^{n-k} y^k \right) \\
&= \sum_{k=0}^{n} \binom{n}{k} x^{n+1-k} y^k + \sum_{k=0}^{n} \binom{n}{k} x^{n-k} y^{k+1} \\
&= \binom{n}{0} x^{n+1} + \sum_{k=1}^{n} \binom{n}{k} x^{n+1-k} y^k + \sum_{k=0}^{n-1} \binom{n}{k} x^{n-k} y^{k+1} + \binom{n}{n} y^{n+1}\\
&= x^{n+1} + \sum_{k=1}^{n} \left[ \binom{n}{k} + \binom{n}{k-1} \right] x^{n+1-k} y^k + y^{n+1}\\
&=x^{n+1} + \sum_{k=1}^{n} \binom{n+1}{k} x^{n+1-k} y^k + y^{n+1}\\
&=\sum_{k=0}^{n+1} \binom{n+1}{k} x^{n+1-k} y^k.
    \end{align*}
    即证.
\end{proof} 
\section{组合恒等式}
通过二项式定理和组合数的性质,可以得到诸多有益的推论和组合恒等式.

\begin{corollary}
    集合\(S\)含有奇数个元素的子集的数目恒等于含有偶数个元素的子集的数目.
\end{corollary}
\begin{proof}
    设\(|S|=n\). 在二项式定理中,令\(x=1,y=-1\),得
    \[\binom{n}{0} - \binom{n}{1} + \binom{n}{2} - \cdots + (-1)^n \binom{n}{n} = 0 .\]

    令\(x=1,y=1\),得
    \[\binom{n}{0} + \binom{n}{1} + \binom{n}{2} + \cdots + \binom{n}{n} = 2^n.\]

    因此有
    \begin{align*}
\binom{n}{0} + \binom{n}{2} + \cdots &= 2^{n-1}, \\
\binom{n}{1} + \binom{n}{3} + \cdots &= 2^{n-1}.
    \end{align*}

    即证.
\end{proof}

对等式进行求导和积分是构造组合恒等式的常见思路.如对等式\((1 + x)^n = \sum_{k=0}^{n} \binom{n}{k} x^k\),两边对\(x\)求导得\(n(1 + x)^{n-1} = \sum_{k=1}^{n} k \binom{n}{k} x^{k-1}\),带入\(x=1\)得\(n 2^{n-1} = \sum_{k=1}^{n} k \binom{n}{k}\).更进一步地,该组合恒等式还可以在两边乘以\(x\)后继续迭代求导,以得到\(\sum_{k=1}^{n} k^p \binom{n}{k}\)对任意常数\(p\)的组合恒等式.一个积分构造组合恒等式的例子:

\begin{corollary}
    $$\frac{2^{n+1}-1}{n+1} = \sum_{k=0}^{n} \binom{n}{k} \frac{1}{k+1}.$$
\end{corollary}
\begin{proof}
    对等式\((1 + x)^n = \sum_{k=0}^{n} \binom{n}{k} x^k\)两边对\(x\)求不定积分:

    \[\frac{(1+x)^{n+1}}{n+1} = \sum_{k=0}^{n} \binom{n}{k} \frac{x^{k+1}}{k+1} + C.\]

    令\(x=0\),解得\(C=\frac{1}{n+1}\). 令\(x=1\),即得所求式子.注意两边求不定积分容易遗漏常数\(C\).
\end{proof}

\begin{corollary}[范德蒙卷积公式]
    对正整数\(m_1, m_2, n\),有
    \[\sum_{k=0}^{n} \binom{m_1}{k} \binom{m_2}{n-k} = \binom{m_1+m_2}{n}.\]
\end{corollary}
\begin{proof}
    设\(S\)是一个有\(m_1+m_2\)个元素的集合,我们需要计数\(S\)的\(n\)子集的数目(即等式右边).把\(S\)划分成\(A,B\)两个子集,其中\(|A|=m_1,|B|=m_2\).利用这个划分去划分\(S\)的\(n\)子集.\(S\)的每一个\(n\)子集包含\(k\)和\(A\)的元素和\(n-k\)和\(B\)的元素,这里\(0\leq k \leq n\).根据加法原理计数所有\(k\)的取值情况,可得
    \[\binom{m_1+m_2}{n}=\sum_{k=0}^{n} \binom{m_1}{k} \binom{m_2}{n-k} .\]
    即证.
\end{proof}

特别地,我们有:
\[\sum_{k=0}^{n} \binom{n}{k}^2 = \binom{2n}{n} \quad (n \geqslant 0).\]

\section{二项式系数的单峰性}
\begin{theorem}
    设 $n$ 为正整数.二项式系数序列
\[
\binom{n}{0}, \binom{n}{1}, \binom{n}{2}, \ldots, \binom{n}{n}
\]
是单峰序列.更精确地说,如果 $n$ 是偶数,则
\[
\binom{n}{0} < \binom{n}{1} < \cdots < \binom{n}{n/2} \]\[
\binom{n}{n/2} > \cdots > \binom{n}{n-1} > \binom{n}{n}.
\]
如果 $n$ 是奇数,则
\[
\binom{n}{0} < \binom{n}{1} < \cdots < \binom{n}{(n-1)/2} = \binom{n}{(n+1)/2} \]\[
\binom{n}{(n+1)/2} > \cdots > \binom{n}{n-1} > \binom{n}{n}.
\]
\end{theorem}
\begin{proof}
    考虑\(1\leq k\leq n\).
    \[\frac{\binom{n}{k}}{\binom{n}{k-1}} = \frac{\frac{n!}{k!(n-k)!}}{\frac{n!}{(k-1)!(n-k+1)!}} = \frac{n-k+1}{k}.\]

    比较\(k\)和\(n-k+1\)的大小关系即可得出结论.
\end{proof}
\begin{corollary}
    对于正整数 $n$,二项式系数
\[
\binom{n}{0}, \binom{n}{1}, \binom{n}{2}, \ldots, \binom{n}{n}
\]
中的最大者为
\[
\binom{n}{\lfloor n/2 \rfloor} = \binom{n}{\lceil n/2 \rceil}.
\]
\end{corollary}

待补充