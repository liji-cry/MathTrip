\chapter{二项式系数}
\section*{引言}
本章内容主要来自Brualdi <<组合数学>> 第5章(二项式系数)的内容及对其的拓展与思考.
\section*{预备知识}
基本组合数公式: 
\[\binom{n}{k} = \binom{n}{n-k}.  \]

帕斯卡公式:
\[\binom{n}{k} = \binom{n-1}{k}+\binom{n-1}{k-1}.  \]

\section{二项式定理}
\begin{theorem}[二项式定理]
    设\(n\)是正整数.  对所有的\(x\)和\(y\),  有\[(x + y)^n = \sum_{k=0}^{n} \binom{n}{k} x^{n-k} y^k.  \]
\end{theorem}
\begin{proof}
    法1: 乘积展开, 对项\(x^{n-k}y^k\), 选择\(k\)个因子为\(y\)而剩下\(n-k\)个因子为\(x\)的组合数.

    法2: 归纳法.  \(n=1\)时等式成立.

    假设等式对\(n\)成立, 对于\(n+1\): 

    \begin{align*}
        (x + y)^{n+1} & = (x + y) \left( \sum_{k=0}^{n} \binom{n}{k} x^{n-k} y^k \right)                                                                          \\
                      & = x \left( \sum_{k=0}^{n} \binom{n}{k} x^{n-k} y^k \right) + y \left( \sum_{k=0}^{n} \binom{n}{k} x^{n-k} y^k \right)                     \\
                      & = \sum_{k=0}^{n} \binom{n}{k} x^{n+1-k} y^k + \sum_{k=0}^{n} \binom{n}{k} x^{n-k} y^{k+1}                                                 \\
                      & = \binom{n}{0} x^{n+1} + \sum_{k=1}^{n} \binom{n}{k} x^{n+1-k} y^k + \sum_{k=0}^{n-1} \binom{n}{k} x^{n-k} y^{k+1} + \binom{n}{n} y^{n+1} \\
                      & = x^{n+1} + \sum_{k=1}^{n} \left[ \binom{n}{k} + \binom{n}{k-1} \right] x^{n+1-k} y^k + y^{n+1}                                           \\
                      & =x^{n+1} + \sum_{k=1}^{n} \binom{n+1}{k} x^{n+1-k} y^k + y^{n+1}                                                                          \\
                      & =\sum_{k=0}^{n+1} \binom{n+1}{k} x^{n+1-k} y^k.
    \end{align*}
    即证.
\end{proof}
\section{组合恒等式}
通过二项式定理和组合数的性质, 可以得到诸多有益的推论和组合恒等式.

\begin{corollary}
    \label{exstl1}
    集合\(S\)含有奇数个元素的子集的数目恒等于含有偶数个元素的子集的数目.
\end{corollary}
\begin{proof}
    设\(|S|=n\).   在二项式定理中, 令\(x=1, y=-1\), 得
    \[\binom{n}{0} - \binom{n}{1} + \binom{n}{2} - \cdots + (-1)^n \binom{n}{n} = 0 .  \]

    令\(x=1, y=1\), 得
    \[\binom{n}{0} + \binom{n}{1} + \binom{n}{2} + \cdots + \binom{n}{n} = 2^n.  \]

    因此有
    \begin{align*}
        \binom{n}{0} + \binom{n}{2} + \cdots & = 2^{n-1}, \\
        \binom{n}{1} + \binom{n}{3} + \cdots & = 2^{n-1}.
    \end{align*}

    即证.
\end{proof}

对等式进行求导和积分是构造组合恒等式的常见思路.  如对等式\((1 + x)^n = \sum_{k=0}^{n} \binom{n}{k} x^k\), 两边对\(x\)求导得\(n(1 + x)^{n-1} = \sum_{k=1}^{n} k \binom{n}{k} x^{k-1}\), 带入\(x=1\)得\(n 2^{n-1} = \sum_{k=1}^{n} k \binom{n}{k}\).  更进一步地, 该组合恒等式还可以在两边乘以\(x\)后继续迭代求导, 以得到\(\sum_{k=1}^{n} k^p \binom{n}{k}\)对任意常数\(p\)的组合恒等式.  一个积分构造组合恒等式的例子: 

\begin{corollary}
    $$\frac{2^{n+1}-1}{n+1} = \sum_{k=0}^{n} \binom{n}{k} \frac{1}{k+1}.  $$
\end{corollary}
\begin{proof}
    对等式\((1 + x)^n = \sum_{k=0}^{n} \binom{n}{k} x^k\)两边对\(x\)求不定积分: 

    \[\frac{(1+x)^{n+1}}{n+1} = \sum_{k=0}^{n} \binom{n}{k} \frac{x^{k+1}}{k+1} + C.  \]

    令\(x=0\), 解得\(C=\frac{1}{n+1}\).   令\(x=1\), 即得所求式子.  注意两边求不定积分容易遗漏常数\(C\).
\end{proof}

\begin{corollary}[范德蒙卷积公式]
    对正整数\(m_1,  m_2,  n\), 有
    \[\sum_{k=0}^{n} \binom{m_1}{k} \binom{m_2}{n-k} = \binom{m_1+m_2}{n}.  \]
\end{corollary}
\begin{proof}
    设\(S\)是一个有\(m_1+m_2\)个元素的集合, 我们需要计数\(S\)的\(n\)子集的数目(即等式右边).  把\(S\)划分成\(A, B\)两个子集, 其中\(|A|=m_1, |B|=m_2\).  利用这个划分去划分\(S\)的\(n\)子集.  \(S\)的每一个\(n\)子集包含\(k\)和\(A\)的元素和\(n-k\)和\(B\)的元素, 这里\(0\leq k \leq n\).  根据加法原理计数所有\(k\)的取值情况, 可得
    \[\binom{m_1+m_2}{n}=\sum_{k=0}^{n} \binom{m_1}{k} \binom{m_2}{n-k} .  \]
    即证.
\end{proof}

特别地, 我们有: 
\[\sum_{k=0}^{n} \binom{n}{k}^2 = \binom{2n}{n} \quad (n \geqslant 0).  \]

\section{二项式系数的单峰性}
\begin{theorem}
    设 $n$ 为正整数.   二项式系数序列
    \[
        \binom{n}{0},  \binom{n}{1},  \binom{n}{2},  \ldots,  \binom{n}{n}
    \]
    是单峰序列.  更精确地说, 如果 $n$ 是偶数, 则
    \[
        \binom{n}{0} < \binom{n}{1} < \cdots < \binom{n}{n/2} \]\[
        \binom{n}{n/2} > \cdots > \binom{n}{n-1} > \binom{n}{n}.
    \]
    如果 $n$ 是奇数, 则
    \[
        \binom{n}{0} < \binom{n}{1} < \cdots < \binom{n}{(n-1)/2} = \binom{n}{(n+1)/2} \]\[
        \binom{n}{(n+1)/2} > \cdots > \binom{n}{n-1} > \binom{n}{n}.
    \]
\end{theorem}
\begin{proof}
    考虑\(1\leq k\leq n\).
    \[\frac{\binom{n}{k}}{\binom{n}{k-1}} = \frac{\frac{n!}{k!(n-k)!}}{\frac{n!}{(k-1)!(n-k+1)!}} = \frac{n-k+1}{k}.  \]

    比较\(k\)和\(n-k+1\)的大小关系即可得出结论.
\end{proof}
\begin{corollary}\label{exsxs1}
    对于正整数 $n$, 二项式系数
    \[
        \binom{n}{0},  \binom{n}{1},  \binom{n}{2},  \ldots,  \binom{n}{n}
    \]
    中的最大者为
    \[
        \binom{n}{\lfloor n/2 \rfloor} = \binom{n}{\lceil n/2 \rceil}.
    \]
\end{corollary}

设$S$是$n$元素集合, 定义$S$的一个反链为一个$S$的子集的集合$\mathcal{A}$, 其中$\mathcal{A}$中的任意元素不存在包含关系.   定义$S$的一个链为一个$S$的子集的集合$\mathcal{C}$, 其中$\mathcal{C}$中的任意两元素之间存在包含关系.   显然, $S$的每一个最大链的长度为$n$, 最大链的数目与$n$元素的排列存在一一对应关系, 为$n!$.

\begin{theorem}[Sperner定理]
    设$S$是$n$元素集合.   那么$S$上的一个反链至多包含$\binom{n}{\lfloor n/2 \rfloor}$个元素.
\end{theorem}
\begin{proof}
    设$\mathcal{A}$是一个反链, $A$是$\mathcal{A}$中的一个元素, $C$是包含$A$的一个最大链. 我们考虑有序对$(A,C)$的数目$\beta $. 每一个$C$包含且只包含$\mathcal{A}$中的一个元素, 因此$\beta \leq n!$. (考虑每个$C$对应的$A$)

    对于$\mathcal{A}$中的一个元素$A$, 设$|A|=k$. 则包含$A$的所有最大链数目为$k!(n-k)!$. (考虑每一个$A$对应的$C$)

    设$\alpha_k$是反链$\mathcal{A}$中大小为$k$的元素个数, 则$|\mathcal{A}|=\sum_{k=0}^{n}\alpha_k$. 于是
    $$\beta=\sum_{k=0}^{n}\alpha_k k!(n-k)!\leq n.$$

    变形得
    $$\sum_{k=0}^{n} \frac{a_k}{\binom{n}{k}} \leq 1.$$

    根据推论\ref{exsxs1}可得
    $$|\mathcal{A}|=\sum_{k=0}^{n}\alpha_k \leq\binom{n}{\lfloor n/2 \rfloor}.$$
    即证.
\end{proof}

\section{多项式定理}
定义多项式系数
\[
    \binom{n}{n_1\; n_2 \cdots n_t} = \frac{n!}{n_1! \, n_2! \cdots n_t!.}
\]
其中,\( n_1\;, n_2, \cdots, n_t \) 是非负整数且$n_1 + n_2 + \cdots + n_t = n$.

\begin{theorem}[多项式系数的帕斯卡公式]\label{dxspsk}
    $$\binom{n}{n_1 \; n_2 \; \cdots \; n_t} = \binom{n-1}{n_1-1 \; n_2 \; \cdots \; n_t} + \binom{n-1}{n_1 \; n_2-1 \; \cdots \; n_t} + \cdots + \binom{n-1}{n_1 \; n_2 \; \cdots \; n_t-1}.$$
\end{theorem}\
对右边式子通分即可得到证明.which perl

\begin{theorem}[多项式定理]
    设 \( n \) 是正整数.  对于所有的 \( x_1, x_2, \cdots, x_t \), 有
    \[
        (x_1 + x_2 + \cdots + x_t)^n = \sum \binom{n}{n_1 \, n_2 \cdots n_t} x_1^{n_1} x_2^{n_2} \cdots x_t^{n_t}.
    \]
\end{theorem}
\begin{proof}
    法1: 根据乘法原理, 式子$x_1^{n_1} x_2^{n_2} \cdots x_t^{n_t}$出现的次数为
    $$\binom{n}{n_1} \binom{n-n_1}{n_2} \cdots \binom{n-n_1-\cdots-n_{r-1}}{n_r}=\frac{n!}{n_1! \, n_2! \cdots n_t!}=\binom{n}{n_1\; n_2 \cdots n_t}.$$即证.

    法2: 归纳法. 注意是对$n$归纳, 而不是对$k$归纳. $n=1$时等式成立.

    若取值为$n$时等式成立, 即$(x_1 + x_2 + \cdots + x_t)^n = \sum \binom{n}{n_1 \, n_2 \cdots n_t} x_1^{n_1} x_2^{n_2} \cdots x_t^{n_t}.$ 当取值为$n+1$时:

    $$(x_1 + x_2 + \cdots + x_t)^{n+1} = \sum_{i=1}^{k}x_i  \sum \binom{n}{n_1 \, n_2 \cdots n_t} x_1^{n_1} x_2^{n_2} \cdots x_t^{n_t}.$$

    考虑$(x_1 + x_2 + \cdots + x_t)^{n+1}$中, 项$x_1^{m_1} x_2^{m_2} \cdots x_t^{m_t}$的系数, 其中$m_1+m_2+\cdots m_t=n+1$. 该项可能由$k$种方式产生:

    \begin{itemize}
        \item 从 \( x_1^{m_1 - 1} x_2^{m_2} \cdots x_k^{m_k} \) 乘以 \( x_1 \)(若 \( m_1 \geq 1 \)), 系数为 \( \frac{n!}{(m_1 - 1)! m_2! \cdots m_k!} \);
        \item 从 \( x_1^{m_1} x_2^{m_2 - 1} \cdots x_k^{m_k} \) 乘以 \( x_2 \)(若 \( m_2 \geq 1 \)), 系数为 \( \frac{n!}{m_1! (m_2 - 1)! \cdots m_k!} \);
        \item \(\cdots\)
        \item 从 \( x_1^{m_1} \cdots x_k^{m_k - 1} \) 乘以 \( x_k \)(若 \( m_k \geq 1 \)), 系数为 \( \frac{n!}{m_1! \cdots (m_k - 1)!} \).

    \end{itemize}

    由定理\ref{dxspsk}, $x_1^{m_1} x_2^{m_2} \cdots x_t^{m_t}$的系数为$\binom{n+1}{m_1 \; m_2 \; \cdots \; m_t}$, 即证.

\end{proof}

特别地, 当$x_1=x_2=\cdots=x_t=1$时, 有
$$t^n=\binom{n}{n_1 \, n_2 \cdots n_t}.$$
